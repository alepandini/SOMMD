% Options for packages loaded elsewhere
\PassOptionsToPackage{unicode}{hyperref}
\PassOptionsToPackage{hyphens}{url}
%
\documentclass[
]{article}
\usepackage{amsmath,amssymb}
\usepackage{lmodern}
\usepackage{iftex}
\ifPDFTeX
  \usepackage[T1]{fontenc}
  \usepackage[utf8]{inputenc}
  \usepackage{textcomp} % provide euro and other symbols
\else % if luatex or xetex
  \usepackage{unicode-math}
  \defaultfontfeatures{Scale=MatchLowercase}
  \defaultfontfeatures[\rmfamily]{Ligatures=TeX,Scale=1}
\fi
% Use upquote if available, for straight quotes in verbatim environments
\IfFileExists{upquote.sty}{\usepackage{upquote}}{}
\IfFileExists{microtype.sty}{% use microtype if available
  \usepackage[]{microtype}
  \UseMicrotypeSet[protrusion]{basicmath} % disable protrusion for tt fonts
}{}
\makeatletter
\@ifundefined{KOMAClassName}{% if non-KOMA class
  \IfFileExists{parskip.sty}{%
    \usepackage{parskip}
  }{% else
    \setlength{\parindent}{0pt}
    \setlength{\parskip}{6pt plus 2pt minus 1pt}}
}{% if KOMA class
  \KOMAoptions{parskip=half}}
\makeatother
\usepackage{xcolor}
\IfFileExists{xurl.sty}{\usepackage{xurl}}{} % add URL line breaks if available
\IfFileExists{bookmark.sty}{\usepackage{bookmark}}{\usepackage{hyperref}}
\hypersetup{
  pdftitle={SOMMD Notebook 1},
  hidelinks,
  pdfcreator={LaTeX via pandoc}}
\urlstyle{same} % disable monospaced font for URLs
\usepackage[margin=1in]{geometry}
\usepackage{color}
\usepackage{fancyvrb}
\newcommand{\VerbBar}{|}
\newcommand{\VERB}{\Verb[commandchars=\\\{\}]}
\DefineVerbatimEnvironment{Highlighting}{Verbatim}{commandchars=\\\{\}}
% Add ',fontsize=\small' for more characters per line
\usepackage{framed}
\definecolor{shadecolor}{RGB}{248,248,248}
\newenvironment{Shaded}{\begin{snugshade}}{\end{snugshade}}
\newcommand{\AlertTok}[1]{\textcolor[rgb]{0.94,0.16,0.16}{#1}}
\newcommand{\AnnotationTok}[1]{\textcolor[rgb]{0.56,0.35,0.01}{\textbf{\textit{#1}}}}
\newcommand{\AttributeTok}[1]{\textcolor[rgb]{0.77,0.63,0.00}{#1}}
\newcommand{\BaseNTok}[1]{\textcolor[rgb]{0.00,0.00,0.81}{#1}}
\newcommand{\BuiltInTok}[1]{#1}
\newcommand{\CharTok}[1]{\textcolor[rgb]{0.31,0.60,0.02}{#1}}
\newcommand{\CommentTok}[1]{\textcolor[rgb]{0.56,0.35,0.01}{\textit{#1}}}
\newcommand{\CommentVarTok}[1]{\textcolor[rgb]{0.56,0.35,0.01}{\textbf{\textit{#1}}}}
\newcommand{\ConstantTok}[1]{\textcolor[rgb]{0.00,0.00,0.00}{#1}}
\newcommand{\ControlFlowTok}[1]{\textcolor[rgb]{0.13,0.29,0.53}{\textbf{#1}}}
\newcommand{\DataTypeTok}[1]{\textcolor[rgb]{0.13,0.29,0.53}{#1}}
\newcommand{\DecValTok}[1]{\textcolor[rgb]{0.00,0.00,0.81}{#1}}
\newcommand{\DocumentationTok}[1]{\textcolor[rgb]{0.56,0.35,0.01}{\textbf{\textit{#1}}}}
\newcommand{\ErrorTok}[1]{\textcolor[rgb]{0.64,0.00,0.00}{\textbf{#1}}}
\newcommand{\ExtensionTok}[1]{#1}
\newcommand{\FloatTok}[1]{\textcolor[rgb]{0.00,0.00,0.81}{#1}}
\newcommand{\FunctionTok}[1]{\textcolor[rgb]{0.00,0.00,0.00}{#1}}
\newcommand{\ImportTok}[1]{#1}
\newcommand{\InformationTok}[1]{\textcolor[rgb]{0.56,0.35,0.01}{\textbf{\textit{#1}}}}
\newcommand{\KeywordTok}[1]{\textcolor[rgb]{0.13,0.29,0.53}{\textbf{#1}}}
\newcommand{\NormalTok}[1]{#1}
\newcommand{\OperatorTok}[1]{\textcolor[rgb]{0.81,0.36,0.00}{\textbf{#1}}}
\newcommand{\OtherTok}[1]{\textcolor[rgb]{0.56,0.35,0.01}{#1}}
\newcommand{\PreprocessorTok}[1]{\textcolor[rgb]{0.56,0.35,0.01}{\textit{#1}}}
\newcommand{\RegionMarkerTok}[1]{#1}
\newcommand{\SpecialCharTok}[1]{\textcolor[rgb]{0.00,0.00,0.00}{#1}}
\newcommand{\SpecialStringTok}[1]{\textcolor[rgb]{0.31,0.60,0.02}{#1}}
\newcommand{\StringTok}[1]{\textcolor[rgb]{0.31,0.60,0.02}{#1}}
\newcommand{\VariableTok}[1]{\textcolor[rgb]{0.00,0.00,0.00}{#1}}
\newcommand{\VerbatimStringTok}[1]{\textcolor[rgb]{0.31,0.60,0.02}{#1}}
\newcommand{\WarningTok}[1]{\textcolor[rgb]{0.56,0.35,0.01}{\textbf{\textit{#1}}}}
\usepackage{graphicx}
\makeatletter
\def\maxwidth{\ifdim\Gin@nat@width>\linewidth\linewidth\else\Gin@nat@width\fi}
\def\maxheight{\ifdim\Gin@nat@height>\textheight\textheight\else\Gin@nat@height\fi}
\makeatother
% Scale images if necessary, so that they will not overflow the page
% margins by default, and it is still possible to overwrite the defaults
% using explicit options in \includegraphics[width, height, ...]{}
\setkeys{Gin}{width=\maxwidth,height=\maxheight,keepaspectratio}
% Set default figure placement to htbp
\makeatletter
\def\fps@figure{htbp}
\makeatother
\setlength{\emergencystretch}{3em} % prevent overfull lines
\providecommand{\tightlist}{%
  \setlength{\itemsep}{0pt}\setlength{\parskip}{0pt}}
\setcounter{secnumdepth}{-\maxdimen} % remove section numbering
\newlength{\cslhangindent}
\setlength{\cslhangindent}{1.5em}
\newlength{\csllabelwidth}
\setlength{\csllabelwidth}{3em}
\newlength{\cslentryspacingunit} % times entry-spacing
\setlength{\cslentryspacingunit}{\parskip}
\newenvironment{CSLReferences}[2] % #1 hanging-ident, #2 entry spacing
 {% don't indent paragraphs
  \setlength{\parindent}{0pt}
  % turn on hanging indent if param 1 is 1
  \ifodd #1
  \let\oldpar\par
  \def\par{\hangindent=\cslhangindent\oldpar}
  \fi
  % set entry spacing
  \setlength{\parskip}{#2\cslentryspacingunit}
 }%
 {}
\usepackage{calc}
\newcommand{\CSLBlock}[1]{#1\hfill\break}
\newcommand{\CSLLeftMargin}[1]{\parbox[t]{\csllabelwidth}{#1}}
\newcommand{\CSLRightInline}[1]{\parbox[t]{\linewidth - \csllabelwidth}{#1}\break}
\newcommand{\CSLIndent}[1]{\hspace{\cslhangindent}#1}
\usepackage[table]{xcolor}
\usepackage{multirow}
\ifLuaTeX
  \usepackage{selnolig}  % disable illegal ligatures
\fi

\title{SOMMD Notebook 1}
\author{}
\date{\vspace{-2.5em}}

\begin{document}
\maketitle

This is A Notebook that explain how to use the SOMMD package to perform
analysis of molecular dynamics simulations.

Load kohonen and SOMMD packages

\begin{Shaded}
\begin{Highlighting}[]
\FunctionTok{library}\NormalTok{(kohonen)}
\NormalTok{devtools}\SpecialCharTok{::}\FunctionTok{load\_all}\NormalTok{(}\StringTok{".."}\NormalTok{)}
\end{Highlighting}
\end{Shaded}

\begin{verbatim}
## i Loading SOMMD
\end{verbatim}

Read simulation files from pdb and xtc files. Since we have more
replicas we will read all the simulations with a for cycle, and append
the replicas with the cat\_trj function.

\begin{Shaded}
\begin{Highlighting}[]
\CommentTok{\#Read the first replica}
\NormalTok{trj }\OtherTok{\textless{}{-}} \FunctionTok{read.trj}\NormalTok{(}\AttributeTok{trjfile=}\StringTok{"../data/Medium\_Dataset/REP\_001.xtc"}\NormalTok{, }\AttributeTok{topfile=}\StringTok{"../data/Medium\_Dataset/ref.pdb"}\NormalTok{)}
\CommentTok{\#Append all other trj files}
\ControlFlowTok{for}\NormalTok{(trj\_file }\ControlFlowTok{in} \FunctionTok{list.files}\NormalTok{(}\StringTok{"../data/Medium\_Dataset/"}\NormalTok{, }\AttributeTok{pattern =} \StringTok{"*.xtc"}\NormalTok{)[}\SpecialCharTok{{-}}\DecValTok{1}\NormalTok{])\{}
\NormalTok{  rep }\OtherTok{\textless{}{-}} \FunctionTok{read.trj}\NormalTok{(}\AttributeTok{trjfile=}\FunctionTok{paste}\NormalTok{(}\StringTok{"../data/Medium\_Dataset/"}\NormalTok{, trj\_file, }\AttributeTok{sep=}\StringTok{""}\NormalTok{), }
                  \AttributeTok{topfile=}\StringTok{"../data/Medium\_Dataset/ref.pdb"}\NormalTok{)}
\NormalTok{  trj }\OtherTok{\textless{}{-}} \FunctionTok{cat\_trj}\NormalTok{(trj, rep)}
\NormalTok{\}}
\end{Highlighting}
\end{Shaded}

\hypertarget{som-training}{%
\subsection{SOM Training}\label{som-training}}

At this point you have to choose a set of descriptors that will be used
for the training of the SOM. SOMMD have a series of built-in function
that facilitate the computation of some common descriptors. In the
present case we want to study an unfolding process, and to describe the
protein conformation during the unfolding the Cbeta pairwise distances
are a valid choice (Motta et al. 2021). Among these distances, the most
interesting ones are those between atoms forming a contact in the native
(folded) conformation. These distances can be selected using the
native\_contacts function.

\begin{Shaded}
\begin{Highlighting}[]
\CommentTok{\#Read reference pdb with native conformation}
\NormalTok{pdb }\OtherTok{\textless{}{-}}\NormalTok{ bio3d}\SpecialCharTok{::}\FunctionTok{read.pdb}\NormalTok{(}\StringTok{"../data/Medium\_Dataset/ref.pdb"}\NormalTok{)}
\CommentTok{\#Select only Cbeta atoms to perform the analysis}
\NormalTok{sele\_atoms }\OtherTok{\textless{}{-}} \FunctionTok{which}\NormalTok{(trj}\SpecialCharTok{$}\NormalTok{top}\SpecialCharTok{$}\NormalTok{elety}\SpecialCharTok{==}\StringTok{"CB"}\NormalTok{)}
\CommentTok{\#Choose only native contacts}
\NormalTok{sele\_dists }\OtherTok{\textless{}{-}} \FunctionTok{native\_contacts}\NormalTok{(}\AttributeTok{struct=}\NormalTok{pdb, }\AttributeTok{distance=}\FloatTok{1.0}\NormalTok{, }\AttributeTok{atoms=}\NormalTok{sele\_atoms)}
\CommentTok{\#Compute distances for SOM training. }
\NormalTok{DIST }\OtherTok{\textless{}{-}} \FunctionTok{calc\_distances}\NormalTok{(trj, }\AttributeTok{MOL2=}\ConstantTok{FALSE}\NormalTok{, }\AttributeTok{sele=}\NormalTok{sele\_dists, }\AttributeTok{atoms=}\NormalTok{sele\_atoms)}
\end{Highlighting}
\end{Shaded}

At this point the computed distances can be used to train the SOM using
the kohonen function. We will train a sheet-shaped (non toroidal) 8x8
SOM with hexagonal neurons.

\begin{Shaded}
\begin{Highlighting}[]
\NormalTok{SOM }\OtherTok{\textless{}{-}}\NormalTok{ kohonen}\SpecialCharTok{::}\FunctionTok{som}\NormalTok{(DIST, }\AttributeTok{grid =} \FunctionTok{somgrid}\NormalTok{(}\DecValTok{8}\NormalTok{, }\DecValTok{8}\NormalTok{, }\StringTok{"hexagonal"}\NormalTok{, }
                                         \AttributeTok{neighbourhood.fct=}\StringTok{"gaussian"}\NormalTok{, }\AttributeTok{toroidal=}\ConstantTok{FALSE}\NormalTok{), }
                    \AttributeTok{dist.fcts=}\StringTok{"euclidean"}\NormalTok{, }\AttributeTok{rlen=}\DecValTok{500}\NormalTok{, }\AttributeTok{mode=}\StringTok{\textquotesingle{}pbatch\textquotesingle{}}\NormalTok{)}
\end{Highlighting}
\end{Shaded}

We can inspect the shape of the SOM looking at the U-Matrix

\begin{Shaded}
\begin{Highlighting}[]
\FunctionTok{plot}\NormalTok{(SOM, }\AttributeTok{type =} \StringTok{\textquotesingle{}dist.neighbours\textquotesingle{}}\NormalTok{, }\AttributeTok{heatkey =} \ConstantTok{TRUE}\NormalTok{, }\AttributeTok{shape=}\StringTok{\textquotesingle{}straight\textquotesingle{}}\NormalTok{, }\AttributeTok{main=}\StringTok{"U{-}Matrix"}\NormalTok{)}
\end{Highlighting}
\end{Shaded}

\includegraphics{SOMMD_files/figure-latex/unnamed-chunk-5-1.pdf}

This plot is telling us regions of the map with high gradient of
difference (yellow/white), and regions that contain neurons similar to
each other (red).

\hypertarget{clustering-of-neurons}{%
\subsection{Clustering of Neurons}\label{clustering-of-neurons}}

To further group neurons similar to each other into clusters an
agglomerative clustering method can be applied to the neuron vectors. As
all the hierarchical clustering methods, one should provide the number
of clusters into which the neurons will be divided. To choose a
reasonable number of clusters one can look at the silhouette profiles:

\begin{Shaded}
\begin{Highlighting}[]
\FunctionTok{par}\NormalTok{(}\AttributeTok{mfrow=}\FunctionTok{c}\NormalTok{(}\DecValTok{1}\NormalTok{,}\DecValTok{2}\NormalTok{))}
\CommentTok{\#Plot the silhouette score}
\FunctionTok{plot.silhouette.score}\NormalTok{(SOM, }\AttributeTok{clust\_method=}\StringTok{"complete"}\NormalTok{, }\AttributeTok{intervall=}\FunctionTok{seq}\NormalTok{(}\DecValTok{2}\NormalTok{,}\DecValTok{30}\NormalTok{))}
\CommentTok{\#Plot the silhouette profile for 8 number of clusters}
\FunctionTok{plot.silhouette.profile}\NormalTok{(SOM, }\AttributeTok{Nclus=}\DecValTok{8}\NormalTok{, }\AttributeTok{clust\_method=}\StringTok{"complete"}\NormalTok{)}
\end{Highlighting}
\end{Shaded}

\includegraphics{SOMMD_files/figure-latex/unnamed-chunk-6-1.pdf}

From this plot one can choose a good value for the number of clusters
and plot the resulting SOM. This should ideally be a maximum in the
average silhouette score plot, but looking at the single silhouette
profiles one can further inspect the quality of the clusters.

\begin{Shaded}
\begin{Highlighting}[]
\CommentTok{\#Divide the SOM in the selected number of clusters}
\NormalTok{SOM.hc }\OtherTok{\textless{}{-}} \FunctionTok{cutree}\NormalTok{(}\FunctionTok{hclust}\NormalTok{(}\FunctionTok{dist}\NormalTok{(SOM}\SpecialCharTok{$}\NormalTok{codes[[}\DecValTok{1}\NormalTok{]], }\AttributeTok{method=}\StringTok{"euclidean"}\NormalTok{), }\AttributeTok{method=}\StringTok{"complete"}\NormalTok{), }\DecValTok{8}\NormalTok{)}
\CommentTok{\#Choose a pleasing to the eye set of colors}
\NormalTok{COL.SCALE }\OtherTok{\textless{}{-}} \FunctionTok{c}\NormalTok{(}\StringTok{"\#1f78b4"}\NormalTok{, }\StringTok{"\#33a02c"}\NormalTok{, }\StringTok{"\#e31a1c"}\NormalTok{, }\StringTok{"\#ffff88"}\NormalTok{, }\StringTok{"\#6a3d9a"}\NormalTok{, }
               \StringTok{"\#a0451f"}\NormalTok{, }\StringTok{"\#96c3dc"}\NormalTok{, }\StringTok{"\#fbb25c"}\NormalTok{, }\StringTok{"\#ff7f00"}\NormalTok{, }\StringTok{"\#bea0cc"}\NormalTok{, }
               \StringTok{"\#747474"}\NormalTok{, }\StringTok{"\#f88587"}\NormalTok{, }\StringTok{"\#a4db77"}\NormalTok{)}
\CommentTok{\#Plot the SOM colored by clusters}
\FunctionTok{plot}\NormalTok{(SOM, }\AttributeTok{type =} \StringTok{"mapping"}\NormalTok{, }\AttributeTok{bgcol=}\NormalTok{COL.SCALE[SOM.hc], }\AttributeTok{col=}\FunctionTok{rgb}\NormalTok{(}\DecValTok{0}\NormalTok{,}\DecValTok{0}\NormalTok{,}\DecValTok{0}\NormalTok{,}\DecValTok{0}\NormalTok{), }\AttributeTok{shape=}\StringTok{\textquotesingle{}straight\textquotesingle{}}\NormalTok{, }\AttributeTok{main=}\StringTok{""}\NormalTok{)}
\FunctionTok{add.cluster.boundaries}\NormalTok{(SOM, SOM.hc, }\AttributeTok{lwd=}\DecValTok{5}\NormalTok{)}
\end{Highlighting}
\end{Shaded}

\includegraphics{SOMMD_files/figure-latex/unnamed-chunk-7-1.pdf}

Many personalizations can be applied on the plotted SOM. For example one
can decide to add the number of neurons:

\begin{Shaded}
\begin{Highlighting}[]
\CommentTok{\#Plot the SOM with neuron numbers}
\FunctionTok{plot}\NormalTok{(SOM, }\AttributeTok{type =} \StringTok{"mapping"}\NormalTok{, }\AttributeTok{bgcol=}\NormalTok{COL.SCALE[SOM.hc], }\AttributeTok{col=}\FunctionTok{rgb}\NormalTok{(}\DecValTok{0}\NormalTok{,}\DecValTok{0}\NormalTok{,}\DecValTok{0}\NormalTok{,}\DecValTok{0}\NormalTok{), }\AttributeTok{shape=}\StringTok{\textquotesingle{}straight\textquotesingle{}}\NormalTok{, }\AttributeTok{main=}\StringTok{""}\NormalTok{)}
\FunctionTok{add.cluster.boundaries}\NormalTok{(SOM, SOM.hc, }\AttributeTok{lwd=}\DecValTok{5}\NormalTok{)}
\FunctionTok{som.add.numbers}\NormalTok{(SOM, }\AttributeTok{scale=}\FloatTok{0.5}\NormalTok{, }\AttributeTok{col=}\StringTok{"black"}\NormalTok{)}
\FunctionTok{som.add.clusters.legend}\NormalTok{(}\AttributeTok{NCLUS=}\DecValTok{8}\NormalTok{, }\AttributeTok{COL.SCALE=}\NormalTok{COL.SCALE)}
\end{Highlighting}
\end{Shaded}

\includegraphics{SOMMD_files/figure-latex/unnamed-chunk-8-1.pdf}

\hypertarget{extraction-of-representative-frames}{%
\subsection{Extraction of representative
frames}\label{extraction-of-representative-frames}}

To inspect the conformations associated to each neuron, a representative
structure can be extracted from the SOM:

\begin{Shaded}
\begin{Highlighting}[]
\CommentTok{\#Get a vector of representative frames for each neuron}
\NormalTok{NEUR\_repres }\OtherTok{\textless{}{-}} \FunctionTok{neur.representatives}\NormalTok{(SOM)}
\FunctionTok{cat}\NormalTok{(}\StringTok{"Frames representatives of neurons: "}\NormalTok{)}
\end{Highlighting}
\end{Shaded}

\begin{verbatim}
## Frames representatives of neurons:
\end{verbatim}

\begin{Shaded}
\begin{Highlighting}[]
\FunctionTok{cat}\NormalTok{(NEUR\_repres)}
\end{Highlighting}
\end{Shaded}

\begin{verbatim}
## 849 1273 3301 926 2563 3791 585 2212 1622 1263 3290 3718 1341 1372 3808 2227 1239 2875 2904 NA 3405 NA 1852 2243 NA 2894 2940 3022 3449 3469 1075 1859 1707 2916 2987 3043 3083 1898 2693 1072 120 1740 NA NA 677 1105 3922 3934 146 199 245 294 711 1135 352 2783 1771 174 220 268 1122 3951 3972 1195
\end{verbatim}

\begin{Shaded}
\begin{Highlighting}[]
\CommentTok{\#Get representatives for each cluster}
\NormalTok{CL\_repres }\OtherTok{\textless{}{-}} \FunctionTok{cluster.representatives}\NormalTok{(SOM, SOM.hc)}
\FunctionTok{cat}\NormalTok{(}\StringTok{"}\SpecialCharTok{\textbackslash{}n\textbackslash{}n}\StringTok{Neurons representatives of each cluster: "}\NormalTok{)}
\end{Highlighting}
\end{Shaded}

\begin{verbatim}
## 
## 
## Neurons representatives of each cluster:
\end{verbatim}

\begin{Shaded}
\begin{Highlighting}[]
\FunctionTok{cat}\NormalTok{(CL\_repres}\SpecialCharTok{$}\NormalTok{neurons)}
\end{Highlighting}
\end{Shaded}

\begin{verbatim}
## 10 13 16 28 38 49 55 51
\end{verbatim}

\begin{Shaded}
\begin{Highlighting}[]
\FunctionTok{cat}\NormalTok{(}\StringTok{"}\SpecialCharTok{\textbackslash{}n\textbackslash{}n}\StringTok{Frames representatives of each cluster: "}\NormalTok{)}
\end{Highlighting}
\end{Shaded}

\begin{verbatim}
## 
## 
## Frames representatives of each cluster:
\end{verbatim}

\begin{Shaded}
\begin{Highlighting}[]
\FunctionTok{cat}\NormalTok{(CL\_repres}\SpecialCharTok{$}\NormalTok{frames)}
\end{Highlighting}
\end{Shaded}

\begin{verbatim}
## 1263 1341 2227 3022 1898 146 352 245
\end{verbatim}

If NA values appear, they are associated to empty neurons/clusters

To extract a conformation saving the pdb structure file simply use the
trj2pdb function:

\begin{Shaded}
\begin{Highlighting}[]
\CommentTok{\#Estract the representative conformation of Neuron 8}
\FunctionTok{trj2pdb}\NormalTok{(}\AttributeTok{traj =}\NormalTok{ trj, }\AttributeTok{frame=}\NormalTok{NEUR\_repres[}\DecValTok{8}\NormalTok{], }\AttributeTok{filename =} \StringTok{"../output/Neuron\_8.pdb"}\NormalTok{)}

\CommentTok{\#Estract the representative conformation of Cluster B}
\FunctionTok{trj2pdb}\NormalTok{(}\AttributeTok{traj =}\NormalTok{ trj, }\AttributeTok{frame=}\NormalTok{CL\_repres}\SpecialCharTok{$}\NormalTok{frames[}\StringTok{"B"}\NormalTok{], }\AttributeTok{filename =} \StringTok{"../output/Cluster\_B.pdb"}\NormalTok{)}
\end{Highlighting}
\end{Shaded}

\hypertarget{representation-of-properties-on-som}{%
\subsection{Representation of properties on
SOM}\label{representation-of-properties-on-som}}

A property of the neurons can be represented using circles with the size
proportional to the value of a property of the neuron. For example one
can use neurons to represent the population of each neuron:

\begin{Shaded}
\begin{Highlighting}[]
\CommentTok{\#Plot the SOM with circles with size proportional to the neuron population }
\FunctionTok{plot}\NormalTok{(SOM, }\AttributeTok{type =} \StringTok{"mapping"}\NormalTok{, }\AttributeTok{bgcol=}\NormalTok{COL.SCALE[SOM.hc], }\AttributeTok{col=}\FunctionTok{rgb}\NormalTok{(}\DecValTok{0}\NormalTok{,}\DecValTok{0}\NormalTok{,}\DecValTok{0}\NormalTok{,}\DecValTok{0}\NormalTok{), }\AttributeTok{shape=}\StringTok{\textquotesingle{}straight\textquotesingle{}}\NormalTok{, }\AttributeTok{main=}\StringTok{""}\NormalTok{)}
\FunctionTok{add.cluster.boundaries}\NormalTok{(SOM, SOM.hc, }\AttributeTok{lwd=}\DecValTok{5}\NormalTok{)}
\NormalTok{POP }\OtherTok{\textless{}{-}} \ConstantTok{NULL}
\ControlFlowTok{for}\NormalTok{(NEURON }\ControlFlowTok{in} \DecValTok{1}\SpecialCharTok{:}\FunctionTok{nrow}\NormalTok{(SOM}\SpecialCharTok{$}\NormalTok{grid}\SpecialCharTok{$}\NormalTok{pts))\{}
\NormalTok{    POP }\OtherTok{\textless{}{-}} \FunctionTok{c}\NormalTok{(POP, }\FunctionTok{length}\NormalTok{(}\FunctionTok{which}\NormalTok{(SOM}\SpecialCharTok{$}\NormalTok{unit.classif}\SpecialCharTok{==}\NormalTok{NEURON)))}
\NormalTok{\}}
\FunctionTok{SOM.add.circles}\NormalTok{(SOM, POP, }\AttributeTok{scale=}\FloatTok{0.9}\NormalTok{)}
\end{Highlighting}
\end{Shaded}

\includegraphics{SOMMD_files/figure-latex/unnamed-chunk-11-1.pdf}

Instead of using circles to plot a property over the plot, you may use
colors-scales. This is usually done to represent the average property of
frames belonging to each neurons. In that case, simply compute the
average property for each neuron and then color the map accordingly.
Here we compute the distance between the Calpha atoms of the C- and N-
terminal ends of the domain and plot this property:

\begin{Shaded}
\begin{Highlighting}[]
\CommentTok{\#Select the index of the first and last CA atoms}
\NormalTok{Terminals }\OtherTok{\textless{}{-}} \FunctionTok{c}\NormalTok{(}\FunctionTok{head}\NormalTok{(}\FunctionTok{which}\NormalTok{(trj}\SpecialCharTok{$}\NormalTok{top}\SpecialCharTok{$}\NormalTok{elety}\SpecialCharTok{==}\StringTok{"CA"}\NormalTok{),}\DecValTok{1}\NormalTok{), }\FunctionTok{tail}\NormalTok{(}\FunctionTok{which}\NormalTok{(trj}\SpecialCharTok{$}\NormalTok{top}\SpecialCharTok{$}\NormalTok{elety}\SpecialCharTok{==}\StringTok{"CA"}\NormalTok{),}\DecValTok{1}\NormalTok{))}
\CommentTok{\#Compute distances between these two atoms in every frame of the simulation}
\NormalTok{Term\_dist }\OtherTok{\textless{}{-}} \FunctionTok{apply}\NormalTok{(trj}\SpecialCharTok{$}\NormalTok{coord[Terminals,,], }\DecValTok{3}\NormalTok{, dist)}
\CommentTok{\#Compute average property value for each neuron}
\NormalTok{Neur.avg.d }\OtherTok{\textless{}{-}} \FunctionTok{average.neur.property}\NormalTok{(SOM, Term\_dist)}
\FunctionTok{par}\NormalTok{(}\AttributeTok{mfrow=}\FunctionTok{c}\NormalTok{(}\DecValTok{1}\NormalTok{,}\DecValTok{2}\NormalTok{))}

\CommentTok{\#Plot the SOM with circles with size proportional to the average distance values}
\FunctionTok{plot}\NormalTok{(SOM, }\AttributeTok{type =} \StringTok{"mapping"}\NormalTok{, }\AttributeTok{bgcol=}\NormalTok{COL.SCALE[SOM.hc], }\AttributeTok{col=}\FunctionTok{rgb}\NormalTok{(}\DecValTok{0}\NormalTok{,}\DecValTok{0}\NormalTok{,}\DecValTok{0}\NormalTok{,}\DecValTok{0}\NormalTok{), }\AttributeTok{shape=}\StringTok{\textquotesingle{}straight\textquotesingle{}}\NormalTok{, }\AttributeTok{main=}\StringTok{""}\NormalTok{)}
\FunctionTok{add.cluster.boundaries}\NormalTok{(SOM, SOM.hc, }\AttributeTok{lwd=}\DecValTok{5}\NormalTok{)}
\FunctionTok{SOM.add.circles}\NormalTok{(SOM, Neur.avg.d, }\AttributeTok{scale=}\FloatTok{0.5}\NormalTok{)}
\CommentTok{\#Plot the SOM colored according to the average property value of each neuron}
\FunctionTok{plot}\NormalTok{(SOM, }\AttributeTok{type =} \StringTok{"property"}\NormalTok{, }\AttributeTok{property=}\NormalTok{Neur.avg.d, }\AttributeTok{shape=}\StringTok{\textquotesingle{}straight\textquotesingle{}}\NormalTok{, }\AttributeTok{palette.name=}\FunctionTok{colorRampPalette}\NormalTok{(}\FunctionTok{c}\NormalTok{(}\StringTok{"blue"}\NormalTok{, }\StringTok{"yellow"}\NormalTok{, }\StringTok{"red"}\NormalTok{)), }\AttributeTok{main=}\StringTok{""}\NormalTok{)}
\FunctionTok{add.cluster.boundaries}\NormalTok{(SOM, SOM.hc, }\AttributeTok{lwd=}\DecValTok{5}\NormalTok{)}
\end{Highlighting}
\end{Shaded}

\includegraphics{SOMMD_files/figure-latex/unnamed-chunk-12-1.pdf}

Any external property can be used (like the pulling force applied during
steered MD) provided that every property value is associated to a frame.

\hypertarget{trace-pathways}{%
\subsection{Trace Pathways}\label{trace-pathways}}

At this point one can trace the pathways sampled during each replica on
the SOM. This can be done using the function trace\_path. This function
draw the pathway followed by a simulation on the SOM. In order to
simplify the plot of pathways from different replicas, the trj object
contains the information of start and end of each replica merged with
cat\_trj in trj\$start and trj\$end. Using this information one can plot
the pathway of a specific replica:

\begin{Shaded}
\begin{Highlighting}[]
\CommentTok{\#Plot the SOM colored by clusters}
\FunctionTok{par}\NormalTok{(}\AttributeTok{mfrow=}\FunctionTok{c}\NormalTok{(}\DecValTok{2}\NormalTok{,}\DecValTok{2}\NormalTok{))}
\ControlFlowTok{for}\NormalTok{(rep }\ControlFlowTok{in} \FunctionTok{c}\NormalTok{(}\DecValTok{1}\NormalTok{,}\DecValTok{3}\NormalTok{,}\DecValTok{6}\NormalTok{,}\DecValTok{8}\NormalTok{))\{}
  \FunctionTok{plot}\NormalTok{(SOM, }\AttributeTok{type =} \StringTok{"mapping"}\NormalTok{, }\AttributeTok{bgcol=}\NormalTok{COL.SCALE[SOM.hc], }\AttributeTok{col=}\FunctionTok{rgb}\NormalTok{(}\DecValTok{0}\NormalTok{,}\DecValTok{0}\NormalTok{,}\DecValTok{0}\NormalTok{,}\DecValTok{0}\NormalTok{), }
       \AttributeTok{shape=}\StringTok{\textquotesingle{}straight\textquotesingle{}}\NormalTok{, }\AttributeTok{main=}\FunctionTok{paste}\NormalTok{(}\StringTok{"Replica "}\NormalTok{, rep, }\AttributeTok{sep=}\StringTok{""}\NormalTok{))}
  \FunctionTok{add.cluster.boundaries}\NormalTok{(SOM, SOM.hc, }\AttributeTok{lwd=}\DecValTok{3}\NormalTok{)}
  \FunctionTok{trace\_path}\NormalTok{(SOM, }\AttributeTok{start=}\NormalTok{trj}\SpecialCharTok{$}\NormalTok{start, }\AttributeTok{end=}\NormalTok{trj}\SpecialCharTok{$}\NormalTok{end, }\AttributeTok{N=}\NormalTok{rep, }\AttributeTok{scale=}\FloatTok{0.5}\NormalTok{)}
\NormalTok{\}}
\end{Highlighting}
\end{Shaded}

\includegraphics{SOMMD_files/figure-latex/unnamed-chunk-13-1.pdf}

The type of sampled pathways can be inspected using a pathway clustering
method:

\begin{Shaded}
\begin{Highlighting}[]
\NormalTok{path.clust }\OtherTok{\textless{}{-}} \FunctionTok{cluster.pathways}\NormalTok{(SOM, }\AttributeTok{start=}\NormalTok{trj}\SpecialCharTok{$}\NormalTok{start, }\AttributeTok{end=}\NormalTok{trj}\SpecialCharTok{$}\NormalTok{end, }\AttributeTok{time.dep=}\StringTok{"dependent"}\NormalTok{)}
\FunctionTok{plot}\NormalTok{(path.clust, }\AttributeTok{xlab=}\StringTok{""}\NormalTok{)}
\end{Highlighting}
\end{Shaded}

\includegraphics{SOMMD_files/figure-latex/unnamed-chunk-14-1.pdf}

This shows the similarity between pathways sampled in different
replicas.

\hypertarget{representing-som-as-a-graph-network}{%
\subsection{\texorpdfstring{\textbf{Representing SOM as a graph
network}}{Representing SOM as a graph network}}\label{representing-som-as-a-graph-network}}

Using the information from the transition matrix of the simulations it
is also possible to build a graph network that represent all the
possible pathways sampled.

\begin{Shaded}
\begin{Highlighting}[]
\CommentTok{\#Compute transition matrix}
\NormalTok{Tr\_mat }\OtherTok{\textless{}{-}} \FunctionTok{Compute\_Transition\_Matrix}\NormalTok{(SOM}\SpecialCharTok{$}\NormalTok{unit.classif, }\AttributeTok{start=}\NormalTok{trj}\SpecialCharTok{$}\NormalTok{start)}
\CommentTok{\#Convert the matrix to an igraph network}
\NormalTok{d }\OtherTok{\textless{}{-}} \FunctionTok{Matrix2Network}\NormalTok{(Tr\_mat)}
\NormalTok{d\_nodiag }\OtherTok{\textless{}{-}} \FunctionTok{Network\_noDiagonal}\NormalTok{(d)}
\NormalTok{net }\OtherTok{\textless{}{-}} \FunctionTok{Network2Graph}\NormalTok{(d\_nodiag, SOM, SOM.hc, COL.SCALE)}
\end{Highlighting}
\end{Shaded}

Using the igraph package several representation of the SOM can be
obtained:

\begin{Shaded}
\begin{Highlighting}[]
\FunctionTok{library}\NormalTok{(igraph)}
\end{Highlighting}
\end{Shaded}

\begin{verbatim}
## 
## Attaching package: 'igraph'
\end{verbatim}

\begin{verbatim}
## The following object is masked from 'package:testthat':
## 
##     compare
\end{verbatim}

\begin{verbatim}
## The following objects are masked from 'package:stats':
## 
##     decompose, spectrum
\end{verbatim}

\begin{verbatim}
## The following object is masked from 'package:base':
## 
##     union
\end{verbatim}

\begin{Shaded}
\begin{Highlighting}[]
\CommentTok{\#edge.start \textless{}{-} ends(net, es=E(net), names=F)[,1]}
\CommentTok{\#edge.col \textless{}{-} V(net)$color[edge.start]}
\CommentTok{\#Plot network with the SOM layout}
\FunctionTok{plot.network.SOM}\NormalTok{(net, SOM, }\AttributeTok{vertex.size=}\FloatTok{1.2}\NormalTok{, }\AttributeTok{edge.size=}\DecValTok{1}\NormalTok{, }\AttributeTok{label=}\ConstantTok{FALSE}\NormalTok{)}
\end{Highlighting}
\end{Shaded}

\includegraphics{SOMMD_files/figure-latex/unnamed-chunk-16-1.pdf}

However, optimal disposition of vertex can be used. Several algorithm
exist able to estimate optimal network vertices placement on the plane.
Here we will use the force-directed layout algorithm by Fruchterman and
Reingold.

\begin{Shaded}
\begin{Highlighting}[]
\NormalTok{coords }\OtherTok{=} \FunctionTok{layout\_with\_fr}\NormalTok{(net, }\AttributeTok{coords=}\NormalTok{SOM}\SpecialCharTok{$}\NormalTok{grid}\SpecialCharTok{$}\NormalTok{pts)}
\FunctionTok{plot}\NormalTok{(net, }\AttributeTok{edge.arrow.size=}\FunctionTok{E}\NormalTok{(net)}\SpecialCharTok{$}\NormalTok{width}\SpecialCharTok{/}\DecValTok{10}\NormalTok{, }\AttributeTok{edge.curved=}\FloatTok{0.17}\NormalTok{, }\AttributeTok{edge.color=}\StringTok{\textquotesingle{}black\textquotesingle{}}\NormalTok{, }\AttributeTok{vertex.label=}\StringTok{""}\NormalTok{, }\AttributeTok{layout=}\NormalTok{coords)}
\end{Highlighting}
\end{Shaded}

\includegraphics{SOMMD_files/figure-latex/unnamed-chunk-17-1.pdf}

Note that since the layout algorithm is not deterministic, you will
obtain a different vertex disposition every time you rerun the
layout\_with\_fr() command. You may also try to run the command several
time to search for a clear vertex disposition.

You may also want to perform a kinetic-like clustering using a community
detection method. Among those available in igraph, the walktrap method
allows to treat biderectional graph. This function tries to find densely
connected subgraphs, also called communities in a graph via random
walks. The idea is that short random walks tend to stay in the same
community.

\begin{Shaded}
\begin{Highlighting}[]
\NormalTok{CL\_walktrap }\OtherTok{\textless{}{-}} \FunctionTok{cluster\_walktrap}\NormalTok{(net)}
\NormalTok{COL }\OtherTok{\textless{}{-}}\NormalTok{ COL.SCALE[}\FunctionTok{membership}\NormalTok{(CL\_walktrap)]}
\FunctionTok{par}\NormalTok{(}\AttributeTok{mfrow=}\FunctionTok{c}\NormalTok{(}\DecValTok{1}\NormalTok{,}\DecValTok{2}\NormalTok{))}
\FunctionTok{plot}\NormalTok{(SOM, }\AttributeTok{type =} \StringTok{"mapping"}\NormalTok{, }\AttributeTok{bgcol=}\NormalTok{COL, }\AttributeTok{col=}\FunctionTok{rgb}\NormalTok{(}\DecValTok{0}\NormalTok{,}\DecValTok{0}\NormalTok{,}\DecValTok{0}\NormalTok{,}\DecValTok{0}\NormalTok{), }\AttributeTok{shape=}\StringTok{\textquotesingle{}straight\textquotesingle{}}\NormalTok{, }\AttributeTok{main=}\StringTok{""}\NormalTok{)}
\FunctionTok{add.cluster.boundaries}\NormalTok{(SOM, SOM.hc, }\AttributeTok{lwd=}\DecValTok{3}\NormalTok{)}
\FunctionTok{plot}\NormalTok{(net, }\AttributeTok{edge.arrow.size=}\FunctionTok{E}\NormalTok{(net)}\SpecialCharTok{$}\NormalTok{width}\SpecialCharTok{/}\DecValTok{10}\NormalTok{, }\AttributeTok{edge.curved=}\FloatTok{0.17}\NormalTok{, }\AttributeTok{vertex.color=}\NormalTok{COL, }\AttributeTok{edge.color=}\StringTok{\textquotesingle{}black\textquotesingle{}}\NormalTok{, }\AttributeTok{vertex.label=}\StringTok{""}\NormalTok{, }\AttributeTok{layout=}\NormalTok{coords)}
\end{Highlighting}
\end{Shaded}

\includegraphics{SOMMD_files/figure-latex/unnamed-chunk-18-1.pdf}

Here neurons are diveded according to the transition probability instead
of their geometric similarity.

To map a property to a graph, simply set the color of each node
according to a colorscale. Start computing a property

\begin{Shaded}
\begin{Highlighting}[]
\CommentTok{\#Select the index of the first and last CA atoms}
\NormalTok{Terminals }\OtherTok{\textless{}{-}} \FunctionTok{c}\NormalTok{(}\FunctionTok{head}\NormalTok{(}\FunctionTok{which}\NormalTok{(trj}\SpecialCharTok{$}\NormalTok{top}\SpecialCharTok{$}\NormalTok{elety}\SpecialCharTok{==}\StringTok{"CA"}\NormalTok{),}\DecValTok{1}\NormalTok{), }\FunctionTok{tail}\NormalTok{(}\FunctionTok{which}\NormalTok{(trj}\SpecialCharTok{$}\NormalTok{top}\SpecialCharTok{$}\NormalTok{elety}\SpecialCharTok{==}\StringTok{"CA"}\NormalTok{),}\DecValTok{1}\NormalTok{))}
\CommentTok{\#Compute distances between these two atoms in every frame of the simulation}
\NormalTok{Term\_dist }\OtherTok{\textless{}{-}} \FunctionTok{apply}\NormalTok{(trj}\SpecialCharTok{$}\NormalTok{coord[Terminals,,], }\DecValTok{3}\NormalTok{, dist)}
\CommentTok{\#Compute average property value for each neuron}
\NormalTok{Neur.avg.d }\OtherTok{\textless{}{-}} \FunctionTok{average.neur.property}\NormalTok{(SOM, Term\_dist)}
\end{Highlighting}
\end{Shaded}

And then set the color of each neuron:

\begin{Shaded}
\begin{Highlighting}[]
\CommentTok{\#Set the color scale}
\NormalTok{col.palette }\OtherTok{\textless{}{-}} \FunctionTok{colorRampPalette}\NormalTok{(}\FunctionTok{c}\NormalTok{(}\StringTok{"blue"}\NormalTok{, }\StringTok{"yellow"}\NormalTok{, }\StringTok{"red"}\NormalTok{))(}\DecValTok{200}\NormalTok{)}
\CommentTok{\#Convert the average property to the colors}
\NormalTok{COL }\OtherTok{\textless{}{-}} \FunctionTok{map2color}\NormalTok{(Neur.avg.d, col.palette)}
\CommentTok{\#Do the plot}
\FunctionTok{plot}\NormalTok{(net, }\AttributeTok{edge.arrow.size=}\FunctionTok{E}\NormalTok{(net)}\SpecialCharTok{$}\NormalTok{width}\SpecialCharTok{/}\DecValTok{10}\NormalTok{, }\AttributeTok{edge.curved=}\FloatTok{0.17}\NormalTok{, }\AttributeTok{vertex.color=}\NormalTok{COL, }\AttributeTok{edge.color=}\StringTok{\textquotesingle{}black\textquotesingle{}}\NormalTok{, }\AttributeTok{vertex.label=}\StringTok{""}\NormalTok{, }\AttributeTok{layout=}\NormalTok{coords)}
\end{Highlighting}
\end{Shaded}

\includegraphics{SOMMD_files/figure-latex/unnamed-chunk-20-1.pdf}

\hypertarget{mapping-of-new-data-on-an-existing-som}{%
\subsection{\texorpdfstring{\textbf{Mapping of new data on an existing
SOM}}{Mapping of new data on an existing SOM}}\label{mapping-of-new-data-on-an-existing-som}}

If you are interested in using a trained SOM to represent a new set of
data (additional simulations not used during the training) you can use
the map.data() function.

\begin{Shaded}
\begin{Highlighting}[]
\CommentTok{\#Read additional simulations}
\NormalTok{trj2 }\OtherTok{\textless{}{-}} \FunctionTok{read.trj}\NormalTok{(}\AttributeTok{trjfile=}\StringTok{"../data/Additional\_Dataset/REP\_011.xtc"}\NormalTok{, }\AttributeTok{topfile=}\StringTok{"../data/Medium\_Dataset/ref.pdb"}\NormalTok{)}
\CommentTok{\#Append all other trj files}
\ControlFlowTok{for}\NormalTok{(trj\_file }\ControlFlowTok{in} \FunctionTok{list.files}\NormalTok{(}\StringTok{"../data/Additional\_Dataset/"}\NormalTok{, }\AttributeTok{pattern =} \StringTok{"*.xtc"}\NormalTok{)[}\SpecialCharTok{{-}}\DecValTok{1}\NormalTok{])\{}
\NormalTok{  rep }\OtherTok{\textless{}{-}} \FunctionTok{read.trj}\NormalTok{(}\AttributeTok{trjfile=}\FunctionTok{paste}\NormalTok{(}\StringTok{"../data/Additional\_Dataset/"}\NormalTok{, trj\_file, }\AttributeTok{sep=}\StringTok{""}\NormalTok{), }
                  \AttributeTok{topfile=}\StringTok{"../data/Medium\_Dataset/ref.pdb"}\NormalTok{)}
\NormalTok{  trj2 }\OtherTok{\textless{}{-}} \FunctionTok{cat\_trj}\NormalTok{(trj2, rep)}
\NormalTok{\}}
\CommentTok{\#Compute distances (the same used for SOM training)}
\NormalTok{DIST2 }\OtherTok{\textless{}{-}} \FunctionTok{calc\_distances}\NormalTok{(trj2, }\AttributeTok{MOL2=}\ConstantTok{FALSE}\NormalTok{, }\AttributeTok{sele=}\NormalTok{sele\_dists, }\AttributeTok{atoms=}\NormalTok{sele\_atoms)}
\CommentTok{\#Map new data on the existing SOM}
\NormalTok{SOM\_new }\OtherTok{\textless{}{-}} \FunctionTok{map.data}\NormalTok{(}\AttributeTok{SOM=}\NormalTok{SOM, }\AttributeTok{X=}\NormalTok{DIST2)}
\end{Highlighting}
\end{Shaded}

At this point you can use the new SOM object (SOM\_new) to generate plot
for the new set of simulations:

\begin{Shaded}
\begin{Highlighting}[]
\CommentTok{\#Plot the SOM colored by clusters}
\FunctionTok{par}\NormalTok{(}\AttributeTok{mfrow=}\FunctionTok{c}\NormalTok{(}\DecValTok{2}\NormalTok{,}\DecValTok{2}\NormalTok{))}
\ControlFlowTok{for}\NormalTok{(rep }\ControlFlowTok{in} \FunctionTok{c}\NormalTok{(}\DecValTok{1}\NormalTok{,}\DecValTok{2}\NormalTok{,}\DecValTok{3}\NormalTok{,}\DecValTok{4}\NormalTok{))\{}
  \FunctionTok{plot}\NormalTok{(SOM\_new, }\AttributeTok{type =} \StringTok{"mapping"}\NormalTok{, }\AttributeTok{bgcol=}\NormalTok{COL.SCALE[SOM.hc], }\AttributeTok{col=}\FunctionTok{rgb}\NormalTok{(}\DecValTok{0}\NormalTok{,}\DecValTok{0}\NormalTok{,}\DecValTok{0}\NormalTok{,}\DecValTok{0}\NormalTok{), }
       \AttributeTok{shape=}\StringTok{\textquotesingle{}straight\textquotesingle{}}\NormalTok{, }\AttributeTok{main=}\FunctionTok{paste}\NormalTok{(}\StringTok{"Additional Replica "}\NormalTok{, rep, }\AttributeTok{sep=}\StringTok{""}\NormalTok{))}
  \FunctionTok{add.cluster.boundaries}\NormalTok{(SOM\_new, SOM.hc, }\AttributeTok{lwd=}\DecValTok{3}\NormalTok{)}
  \FunctionTok{trace\_path}\NormalTok{(SOM\_new, }\AttributeTok{start=}\NormalTok{trj2}\SpecialCharTok{$}\NormalTok{start, }\AttributeTok{end=}\NormalTok{trj2}\SpecialCharTok{$}\NormalTok{end, }\AttributeTok{N=}\NormalTok{rep, }\AttributeTok{scale=}\FloatTok{0.5}\NormalTok{)}
\NormalTok{\}}
\end{Highlighting}
\end{Shaded}

\includegraphics{SOMMD_files/figure-latex/unnamed-chunk-22-1.pdf}

\begin{Shaded}
\begin{Highlighting}[]
\CommentTok{\#Plot the SOM with circles with size proportional to the neuron population }
\FunctionTok{plot}\NormalTok{(SOM\_new, }\AttributeTok{type =} \StringTok{"mapping"}\NormalTok{, }\AttributeTok{bgcol=}\NormalTok{COL.SCALE[SOM.hc], }\AttributeTok{col=}\FunctionTok{rgb}\NormalTok{(}\DecValTok{0}\NormalTok{,}\DecValTok{0}\NormalTok{,}\DecValTok{0}\NormalTok{,}\DecValTok{0}\NormalTok{), }\AttributeTok{shape=}\StringTok{\textquotesingle{}straight\textquotesingle{}}\NormalTok{, }\AttributeTok{main=}\StringTok{""}\NormalTok{)}
\FunctionTok{add.cluster.boundaries}\NormalTok{(SOM\_new, SOM.hc, }\AttributeTok{lwd=}\DecValTok{5}\NormalTok{)}
\NormalTok{POP }\OtherTok{\textless{}{-}} \ConstantTok{NULL}
\ControlFlowTok{for}\NormalTok{(NEURON }\ControlFlowTok{in} \DecValTok{1}\SpecialCharTok{:}\FunctionTok{nrow}\NormalTok{(SOM\_new}\SpecialCharTok{$}\NormalTok{grid}\SpecialCharTok{$}\NormalTok{pts))\{}
\NormalTok{    POP }\OtherTok{\textless{}{-}} \FunctionTok{c}\NormalTok{(POP, }\FunctionTok{length}\NormalTok{(}\FunctionTok{which}\NormalTok{(SOM\_new}\SpecialCharTok{$}\NormalTok{unit.classif}\SpecialCharTok{==}\NormalTok{NEURON)))}
\NormalTok{\}}
\FunctionTok{SOM.add.circles}\NormalTok{(SOM, POP, }\AttributeTok{scale=}\FloatTok{0.9}\NormalTok{)}
\end{Highlighting}
\end{Shaded}

\includegraphics{SOMMD_files/figure-latex/unnamed-chunk-23-1.pdf}

Add a new chunk by clicking the \emph{Insert Chunk} button on the
toolbar or by pressing \emph{Ctrl+Alt+I}.

\hypertarget{refs}{}
\begin{CSLReferences}{1}{0}
\leavevmode\vadjust pre{\hypertarget{ref-Motta2021}{}}%
Motta, Stefano, Alessandro Pandini, Arianna Fornili, and Laura Bonati.
2021. {``Reconstruction of ARNT PAS-B Unfolding Pathways by Steered
Molecular Dynamics and Artificial Neural Networks.''} \emph{Journal of
Chemical Theory and Computation} 17 (4): 2080--89.
\url{https://doi.org/10.1021/acs.jctc.0c01308}.

\end{CSLReferences}

\end{document}
